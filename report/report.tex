\documentclass[a4paper,11pt]{article}

\usepackage[T1]{fontenc}
\usepackage[active]{srcltx}
\usepackage[english,swedish]{babel}
\usepackage[utf8]{inputenc}
\usepackage{amsmath}
\usepackage{amssymb}
\usepackage{graphicx}
\usepackage{listings}
\usepackage{parskip}
\usepackage{underscore}
\usepackage{url}

\urlstyle{same}

\newcommand{\trans}[2][eng.]{(#1 \emph{#2})}
\newcommand{\m}[1]{\boldsymbol{#1}}
\renewcommand{\*}[0]{\cdot}
\renewcommand{\Pr}[1]{\mathbf{Pr}[#1]}
\renewcommand{\v}[1]{\vec{#1}}

\title{
    {\sc DD2488 Kompilatorkonstruktion} \\ 
    Projektrapport
}
\author{
Joel Pettersson \\
joelpet@kth.se \\
\and
Linus Wallgren \\
linuswa@kth.se \\
}
\date{\today}

\begin{document}
\maketitle
\newpage

\section*{Inledning}

En kompilator är ett omfattande program vars huvudsakliga uppgift är att
översätta programspråk till körbar maskinkod. I den här rapporten presenteras
hur vi gick tillväga för att konstruera vår egen kompilator för programspråket
MiniJava; vi beskriver kompilatorns kodstruktur och diskuterar bland annat
designval och dess konsekvenser samt problem som vi har stött på under arbetets
gång.

\section*{Metod}

Vi utgick ifrån specifikationen av MiniJava som ges i \cite{appel2002} med
några små modifikationer av grammatiken. Vårt första steg var att välja ett
verktyg för att generera en lexer och en parser, eftersom det är en relativt
omständlig uppgift att skriva sådana själv från grunden, vilket dessutom är
förknippat med en viss risk för ödesdigra misstag. Vi hade i praktiken att
välja på JFlex (lexergenerator) kombinerat med Java CUP (parsergenerator) samt
JavaCC där den senare kombinerar lexer- och parsergenereringen. Valet föll
slutligen på det första alternativet då Java CUP är en LALR(1)-parser med
vilken varje rimligt programmeringsspråk kan analyseras, enligt Appel
\cite{appel2002}, och således är vanligt förekommande i dessa sammanhang.

Därefter inledde vi arbetet med att skriva en lexerspecifikation för JFlex.
Till slut fick vi ut en lexer som kompilerade, men som vi inte riktigt kunde
sätta in i ett sammanhang och därmed heller inte kunde testa.  Arbetet
fortskred ändå genom att vi leta upp några exempel på Java CUP-specifikationer
som kunde anpassas till MiniJava. När det var klart insåg vi hur kopplingen
mellan lexern och parsern skulle se ut.

Nästa milstolpe bestod av att infoga semantisk åtgärdskod \trans{semantic
action code} i parsern för att bygga ett abstrakt syntaxträd. Det syntaxträdet
-- uppbyggt av klasserna i \texttt{syntaxtree}-paketet -- använde vi senare vid
typkontrollen och därefter vid JVM-bytekodgenereringen. Typkontrollen krävde en
hel del tillägg av kod, men var ändå relativt rättfram. Därefter återstod
alltså själva kodgenereringen, vilken vi gjorde utgående ifrån syntaxträdet med
vissa tillägg. Slutligen återstod endast att kontrollera att allting fungerade,
vilket det naturligtvis inte gjorde på första försöket, men (förhoppningsvis)
väl efter idoga testande och buggletande.

\section*{Resultat}

I denna del beskriver vi den färdiga kompilatorns uppbyggnad och vilka
designval vi har gjort.

\subsection*{Lexer (symbolanalys)}

Lexern är den del av kompilatorn som styckar upp indata, det vill säga
programkoden, till en ström av symboler \trans{tokens}. Den lexer som används i
vår kompilator är, som nämnts ovan, genererad av JFlex och specificeras i
\texttt{src/lex/minijava.lex}. Det mesta i specifikationen är väldigt rättframt
då det i stort sett endast handlar om att beskriva vilka reguljära uttryck som
ska knytas till vilken symbol. Något som ändå vållade vissa bekymmer rörde
hanteringen av negativa tal.

Lexern har inte tillräcklig kunskap för att besluta om ett minustecken framför
ett tal är en del av symbolen eller om det är en egen symbol -- operatorn för
subtraktion. Det innebär att vi inte kan veta om ett tal är negativt eller
positivt. Det för i sin tur med sig ett problem då det går att representera tal
vars absolutvärde är större om talet är negativt än om det är positivt. Det
vill säga även om $-2^{31}$ går att representera med ett 32-bitars heltal går
det inte att representera $2^{31}$; talet kommer alltså bara vara giltigt om
det är negativt, vilket vi konstaterat inte går att bedöma i lexern. Av denna
anledning hanterar lexern tal som strängar, och överlämnar problemet med
negativa tal till parsern.

\subsection*{Parser (syntaxanalys)}

Syntaxanalysen behandlar de symboler som genererats av lexern och matchar dem
mot grammatiken som består av en samling fördefinierade regler, eller
produktioner \trans{productions}. Produktionerna i sig representerar syntaxen
för språket och kommer i slutändan motsvara noder i det syntaxträd som är
grunden för resterande del av kompilatorn.

Java CUP är en LALR(1)-parser och är således relativt kraftfull jämfört med
till exempel en LR(0)- eller LL(0)-parser, vilket i praktiken innebar att
reglerna mer eller mindre direkt motsvarar grammatiken som vi utgick ifrån.

\subsection*{Besökarmönstret}

För att undvika alltför många \texttt{instanceof} i koden använder vi oss av
ett så kallat besökarmönster \trans{visitor pattern} då vi traverserar
syntaxträ. Besökarmönstret så som vi implementerar det resulterar mer eller
mindre i en djupet-först-sökning i syntaxträdet. Sökningen initieras genom att
rotnoden via \texttt{accept(Visitor~v)} bjuder in besökaren som sedan direkt
ges tillbaka kontrollen genom att den accepterande noden (rotnoden, i fallet
med det initierande besöket) anropar \texttt{v.visit(this)}. I
\texttt{Visitor}-klassen finns en \texttt{visit}-metod för vardera nodtyp.  På
det viset kan vi i varje nod utföra lämpliga uppgifter innan vi fortsätter
nedåt genom att anropa \texttt{accept(Visitor~v)} på den aktuella nodens barn.
Exempelvis kan vi skriva ut hur syntaxträdet ser ut genom att låta en besökare
i varje nod skriva ut namnet på noden.

\subsection*{Typgenomgång}

För att generera en karta över vilka identifierare som existerar i koden
användes också besökarmönstret. Genom att gå igenom koden och vid varje
deklaration, såsom variabel klass och metoddeklarationer, sparade vi undan
varje identifierare och dess typ. Datastrukturen som användes är i stort ett
träd, där varje nod representerar ett omfång \trans{scope}. Tanken är att varje
omfång ska ha en koppling mellan identifieraren och typen den innehåller.
Problemet som uppstår då är metoder, vilka inte riktigt har samma regler som
vanliga variabler på grund av överlagring. Med anledning av det hanterar vi
metoderna speciellt, medan vanliga identifierare, såsom variabler, sparas i en
enkel \texttt{HashMap<String, Type>}.

Typ-genomgången ser till att inte samma identifierare deklareras flera gånger
inom samma omfång, och även i vissa fall i underliggande nästlade omfång, såsom
exempelvis inom ett nytt block.

\subsection*{Typkontroll}

Med hjälp av typkopplingen ser vi till att alla tilldelningar och liknande har
samma typ i högerled som i vänsterled. Även typkontrollen görs via
besöksmönstret med skillnaden att varje steg i trädet returnerar en typen av
den noden. På så vis kan man evaluera vilken typ ett komplext uttryck är på ett
simpelt vis.

För att hitta alla typ-fel i koden antas hela tiden att varje operation är
lyckad, vilket innebär att även om typ-kollen upptäcker ett fel någonstans,
returnerar den som om det vore lyckat, för att kunna fortsätta typkontrollen.
Det innebär exempelvis att om indata försöker multiplicera två strängar
rapporterar vi felet men påstår ändå att multiplikationen resulterade i ett
heltal, vilket är att förvänta av en multiplikation.

\subsection*{Kodgenerering}

Kodgenereringen sker via besökarmönstret. I var nod genereras instruktioner som
sparas i en lista. Varje instruktion innehåller information om vilken bytecode
den skall generera samt dess förändring av storleken på operandstacken. Listan
med instruktioner gör det enklare att gå igenom och beräkna hur stor gränsen på
stacken för var metod skall vara, vilket beräknas efter att all kod genererats.
När genereringen är färdig gås listan igenom och varje instruktion skrivs ut
till rätt fil, motsvarande klassen instruktionerna befinner sig i.

I varje metod krävs det att alla lokala variabler har ett unikt id. Varje gång
en variabel tilldelas slås ID för den identifieraren upp i en karta tillhörande
typkopplingen. Ifall variabeln inte tidigare slagits upp genereras ett nytt ID.
På så vis får vi en lat användning av variabler vilket innebär att
variabeldeklarationerna i sig inte påverkar assemblerkoden vilket i sin tur
innebär att icke tilldelade variabler aldrig blir inkluderade i programmet.

Likt identifierarna för de lokala variablerna behövs unika identifierare för
alla etiketter \trans{labels}, som används för hopp \trans{branches}. En klass
med statiska metoder håller koll på de senaste genererade etiketterna för att
undvika krockar.

\section*{Diskussion}

Ett av problemen vi stötte på var hur negativa tal skulle hanteras. Det
naturliga är att låta lexern representera dem som tal och låta minustecknet
vara en del av talet. Det är tvetydligt ifall minustecknet är en del av talet
eller en operator. På grund av detta behövs antingen striktare regler i
grammatiken eller en smartare parser som kan hantera den extra komplexiteten.
Striktare regler innebär helt enkelt att man skiljer de två fallen åt,
exempelvis genom att alltid definiera negativa tal med ett minustecken direkt
följt av talet, medan operatorn har ett mellanrum. Likaså kan man göra parsern
inteligentare genom att låta den hantera operationer utan explicit operator,
utan att parsern får härleda operatorn ifall den högra operanden är negativ.
I båda fallen uppstår det problem när man har tal vid gränsen av vad den
bakomliggande datatypen kan hantera, iom att negativa och positiva tal har
olika stora maximala absolutvärden.

Tack vare att kompilatorn är konstruerad för MiniJava har vi haft möjligheten
att begränsa komplexiteten i kompilatorn eftersom språket är såpass begränsat.
Detta har inneburit allt ifrån att det är få primitiva datatyper att
implementera till att endast behöva implementera en tom standardkonstruktor
samt att vi inte var tvungna att hantera klassers eller deras medlemmars
synlighet. Problemet med begränsningarna i MiniJava har å andra sidan inneburit
att det är klurigt att skriva testfall, då många vanliga programkonstruktioner
inte fungerar.

Eftersom vi i skrivande stund ännu inte har implementerat någon annan backend
än JVM finns det heller inga IR-träd, flödesgrafer eller liknande
implementerat. Det är nästa steg på vägen till att implementera fler backends.
Dock har alla JVM-instruktioner extraherats till ett eget paket för att inte
interferera med eventuella extra backends.

Sammanfattningsvis har vi genom olika redan tillgängliga verktyg, på förhand
given kod och en hel del egna tillägg konstruerat en egen kompilator för det
lilla programspråket MiniJava. Det har givit oss en bättre förståelse för hur
programspråk i allmänhet är uppbyggda och kompilatorns viktiga uppgift. Framför
allt har det här arbetet fått oss att inse vikten av god modularisering vid
större projekt som detta. Eftersom vi redan från början leddes in på rätt spår
lyckades vi, i vår mening, relativt väl med uppdelningen i olika moduler, så
som lexer, parser och semantikanalyserare, även om paketstrukturen hade mått
bra av ytterligare handpåläggning.


\appendix
\section{Kodstruktur}
\subsection*{jvm.jasmin}
Innehåller all kod direkt relaterade till generering av Jasmin-kod, vilket mer
eller mindre är bytecode.
\subsubsection*{jvm.jasmin.directives}
Innehåller Jasmin-specifik kod för att hjälpa Jasmin att generera bytekoden.
\subsubsection*{jvm.jasmin.instructions}
Majoriteten av klasserna i detta paket är bytekod, med ett fåtal undantag och
förenklingar som finns tillgängliga i Jasmin.
\subsection*{lex}
Innehåller alla klasser genererade av lexern, JFlex.
\subsection*{mjc}
Den huvudsakliga ingångspunkten för kompilatorn som ser till att alla delar
blir exekverade och hanterar eventuella fel. Klasserna i paketet har alla en
main-metod för att vara exekverbara.
\subsection*{parser}
Paketet parser innehåller alla klasser genererade av parsern, för tillfället
Java CUP.
\subsection*{syntaxtree}
Alla noder i syntaxträdet har en motsvarande klass i paketet syntaxtree. De
instansieras av parsern vid parsning och används sedan för resterande steg av
kompilatorn.
\subsection*{syntaxtree.visitor}
Alla besökare som använder besökarmönstret finns i paketet syntaxtree.visitor.
Även klasser direkt använda av besökarna befinner sig här. Besökarna använder
sig av en djupet-först-sökning för att gå igenom syntaxträdet.

Viktiga klasser och interfaces är:
\begin{itemize}
\texttt{Visitor} - Ett generellt interface för alla besökare.
\texttt{DepthFirstVisitor} - En implementation av en djupet-först-sökning som används av majoriteten av besökare.
\texttt{SymbolTableVisitor} - Går med hjälp av DepthFirstVisitor igenom trädet och registrerar alla varibeldeklarationer
\texttt{TypeVisitor} - Går igenom syntaxträdet och kollar så att inga typfel finns.
\texttt{CodeGeneratorVisitor} - Genererar den slutgiltiga koden från syntaxträdet med hjälp av DepthFirstVisitor
\end{itemize}

\begin{thebibliography}{99}

    \bibitem{appel2002}
        Appel, A.W. and Palsberg, J.
        (2002).
        \emph{Modern compiler implementation in Java,}.
\end{thebibliography}

\end{document}
